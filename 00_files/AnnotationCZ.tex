%%Nevynechávat volný řádek

\begin{hyphenrules}{nohyphenation}

Laser shock peening je proces povrchové úpravy používaný ke zlepšení  fyzikálně-chemických vlastností (únavová životnost, odolnost proti korozi) kovových součástí. Laser shock peening vnáší pod povrch kovových materiálů zbytková napětí. Aplikace procesu laser shock peening v posledních letech rostou, a to především díky stále rostoucím energiím a klesajícím cenám laserových systémů s parametry vhodnými pro tento proces. Tato práce se snaží vyřešit problém, jak lze software RoboDK a jeho Python application user interface efektivně využít k vytváření programů robotických ramen pro proces laser shock peening. Problém je vyřešen úpravou stávajícího postprocesoru RoboDK. Vytvořené řešení umožňuje vytvářet programy pro robotická ramena speciálně uzpůsobené pro proces laser shock peening. Hlavní přínos práce je zjednodušení vytváření programů robotických ramen pro díly se složitou geometrií. 

\end{hyphenrules}

